% main.tex
\documentclass{mcmthesis}
\mcmsetup{CTeX = false, tcn = 1234567} % 填好你们的控制号
\usepackage{palatino}
\usepackage{lipsum} % 其他你们需要的宏包在这里加

\title{The Title of Your Paper}

\begin{document}

% === 摘要部分 ===
\begin{abstract}
    % 注意:把摘要文字写在 sections/abstract.tex 里
    % 这里只留一行 input
    %!TEX root = ../main.tex
\section{Abstract}
This is our abstract test.
\end{abstract}

\maketitle

% === 正文部分 ===
% 目录(如果比赛要求有的话)
\tableofcontents 
\newpage

% 模块化引入各章节
%!TEX root = ../main.tex
\section{Introduction}
This is intro!




       % 问题背景、重述
%!TEX root = ../main.tex
\section{Assumptions} % 假设与符号说明
\input{sections/3_model_build} % 模型建立
\input{sections/4_solution}    % 模型求解
\input{sections/5_sensitivity} % 灵敏度分析
\input{sections/6_conclusion}  % 结论与优缺点

% === 参考文献 ===
% 建议使用 BibTeX,这样管理参考文献最方便
\bibliographystyle{plain}
\bibliography{references} 

% === 附录 (代码) ===
\begin{appendices}
    \input{sections/appendix_code}
\end{appendices}

\end{document}