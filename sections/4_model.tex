%!TEX root = ../main.tex
\section{Analysis Before Modeling}
% 3.1:电梯情况介绍-------------------------------
\subsection{Overall Analysis of Space Elevator} 
A space elevator is a tether--climber transport system. A high-strength
 tether is anchored at an equatorial ground station and extends through 
 the geostationary region to a counterweight, so that the combined gravity 
 and centrifugal effects keep the tether under tension and approximately radial. 
 Cargo is loaded into a climber and hauled upward along the tether to the apex anchor 
 (the release node specified in the problem). At the apex anchor, the payload is 
 transferred/released onto a translunar trajectory; after a small insertion/correction 
 maneuver if needed, it travels through cislunar space and is finally captured for delivery 
 to the target lunar orbit or the lunar surface. In our model, this process is abstracted as: 
 ground loading $\rightarrow$ tether ascent $\rightarrow$ apex release/transfer $\rightarrow$ Earth--Moon 
 transfer $\rightarrow$ lunar capture/delivery.
 %------------------------------------
 %3.1.1: 物资燃料比的物理解释
 %-------------------------------------

\subsubsection{Fuel-to-cargo ratio under ideal assumptions} 
We quantify the ``hidden'' propellant cost after releasing cargo at the apex anchor. Under the
 ideal assumptions in the prompt (ignore climber mass; deliver cargo to lunar orbit and return)
 , the main steps are: (i) determine the state at release from the apex anchor; (ii) compute 
 the required $\Delta v$ to be captured into low lunar orbit (LLO) and to return; (iii) 
 convert total $\Delta v$ into propellant via the rocket equation.

\paragraph{Step 0: constants and radii.}
Let $\mu_E$ and $\mu_M$ be the Earth and Moon gravitational parameters, and $\omega_E$ be 
Earth spin rate. Define
\begin{equation}
r_{\text{apex}} = R_E + h_{\text{apex}},\qquad r_{\text{moon}} \approx 384{,}400\,\mathrm{km},
\qquad r_{\text{LLO}} = R_M + 100\,\mathrm{km}.
\end{equation}

\paragraph{Step 1: release speed at the apex anchor.}
Because the apex anchor co-rotates with Earth, the tangential release speed (in an inertial 
frame) is
\begin{equation}
v_{\text{release}} = \omega_E\, r_{\text{apex}}.
\end{equation}

\paragraph{Step 2: transfer to the Moon (Earth-centered energy).}
Using conservation of specific mechanical energy for the Earth-centered leg,
\begin{equation}
\frac{v^2}{2} - \frac{\mu_E}{r} = \text{const},
\end{equation}
we obtain the spacecraft speed when it reaches the lunar distance $r_{\text{moon}}$:
\begin{equation}
\frac{v_{\text{release}}^2}{2} - \frac{\mu_E}{r_{\text{apex}}}
= \frac{v_{\text{cross}}^2}{2} - \frac{\mu_E}{r_{\text{moon}}}
\quad\Longrightarrow\quad
v_{\text{cross}} = \sqrt{v_{\text{release}}^2 - 2\mu_E\left(\frac{1}{r_{\text{apex}}}-\frac{1}{r_{\text{moon}}}\right)}.
\end{equation}
The hyperbolic excess speed relative to the Moon upon arrival is denoted $v_{\infty,\text{arr}}$ (obtained from the 
relative velocity geometry in the prompt).

\paragraph{Step 3: lunar capture into LLO (LOI burn).}
For a hyperbolic approach with excess speed $v_{\infty,\text{arr}}$, the periapsis speed at $r_{\text{LLO}}$ is
\begin{equation}
v_p = \sqrt{v_{\infty,\text{arr}}^2 + \frac{2\mu_M}{r_{\text{LLO}}}}.
\end{equation}
The circular speed at $r_{\text{LLO}}$ is
\begin{equation}
v_{\text{circ}} = \sqrt{\frac{\mu_M}{r_{\text{LLO}}}}.
\end{equation}
Thus the lunar-orbit-insertion requirement is
\begin{equation}
\Delta v_{\text{LOI}} = v_p - v_{\text{circ}}.
\end{equation}

\paragraph{Step 4: return to the apex and velocity match.}
After payload drop in LLO, the vehicle performs a trans-Earth injection (TEI) and 
then matches the apex anchor speed at Earth return. Denote these as $\Delta v_{\text{TEI}}$ and $\Delta v_{\text{match}}$ 
(computed from the corresponding vis-viva/energy relations in the prompt).

\paragraph{Step 5: total $\Delta v$ and propellant requirement (Tsiolkovsky).}
The total required increment is
\begin{equation}
\Delta v_{\text{tot}} = \Delta v_{\text{LOI}} + \Delta v_{\text{TEI}} + \Delta v_{\text{match}}.
\end{equation}
For a constant specific impulse $I_{sp}$, the effective exhaust velocity is
\begin{equation}
v_e = I_{sp} g_0.
\end{equation}
The Tsiolkovsky rocket equation gives
\begin{equation}
\frac{m_0}{m_f} = \exp\!\left(\frac{\Delta v_{\text{tot}}}{v_e}\right),
\end{equation}
so the propellant-to-cargo ratio is
\begin{equation}
\frac{m_{\text{fuel}}}{m_{\text{cargo}}} = \frac{m_0-m_f}{m_f} = \exp\!\left(\frac{\Delta v_{\text{tot}}}{v_e}\right)-1.
\end{equation}
% 这里的450是氢氧燃料发动机的比冲,应该在更早的地方就说明好。
Using the 2050 technology parameter $I_{sp}=450\,\mathrm{s}$ (so $v_e\approx 4.41\,\mathrm{km/s}$) 
and the $\Delta v$ budget derived from the above steps in the prompt, we obtain $m_0/m_f\approx 5.2$, hence
\begin{equation}
\frac{m_{\text{fuel}}}{m_{\text{cargo}}} \approx 5.2 - 1 = 4.2.
\end{equation}
Therefore, in the ideal case, sending $1$ unit mass of cargo to the Moon via the space 
elevator requires carrying about $4.2$ units of propellant for the post-release transfer, 
capture, and return.
%------------------------------------------
%3.1.2 电梯的运输电费总计
%------------------------------------------
\subsubsection{Cost of working Space Elevator }
%%%% 此处内容解释还没有改,ai写的
To evaluate the economic feasibility of the Space Elevator system, we calculate the energy 
required to lift a unit mass ($m = 1\,\text{kg}$) from the Earth's surface to the Apex Anchor. 
The dynamics of a climber on the space elevator are governed by the effective potential energy,
 $V_{\text{eff}}(r)$, which accounts for both the Earth's gravitational potential and the 
 centrifugal potential due to the Earth's rotation. The effective potential at a distance $r$ 
 from the Earth's center is given by:

\begin{equation}
    V_{\text{eff}}(r) = -\frac{GM}{r} - \frac{1}{2}\omega^2 r^2
\end{equation}

where:
\begin{itemize}
    \item $G \approx 6.674 \times 10^{-11} \, \text{m}^3\text{kg}^{-1}\text{s}^{-2}$ is the gravitational constant.
    \item $M \approx 5.972 \times 10^{24} \, \text{kg}$ is the mass of the Earth.
    \item $\omega \approx 7.292 \times 10^{-5} \, \text{rad/s}$ is the Earth's angular velocity.
\end{itemize}

Although the Apex Anchor is located at an altitude of 100,000 km, the energy cost is dominated
 by the work required to climb against the Earth's gravity well up to the Geosynchronous Orbit
  (GEO). At GEO ($r_{\text{geo}} \approx 42,164 \, \text{km}$), the gravitational force is 
  exactly balanced by the centrifugal force, representing the point of maximum effective 
  potential energy along the tether. Beyond GEO, the centrifugal force dominates, allowing the 
  climber to theoretically "slide" outward to the Apex. Therefore, the conservative energy 
  requirement is the work done to reach this potential peak.

The effective potential at the Earth's surface ($R_E \approx 6,378 \, \text{km}$) and at GEO are calculated as follows:

\begin{equation}
    V_{\text{eff}}(R_E) \approx -62.50 \, \text{MJ/kg}
\end{equation}
\begin{equation}
    V_{\text{eff}}(r_{\text{geo}}) \approx -14.08 \, \text{MJ/kg}
\end{equation}

The specific work $\Delta E$ required to lift the payload is the difference in potential:

\begin{equation}
    \Delta E = V_{\text{eff}}(r_{\text{geo}}) - V_{\text{eff}}(R_E) \approx 48.42 \, \text{MJ/kg}
\end{equation}

Converting this energy into kilowatt-hours ($1 \, \text{kWh} = 3.6 \, \text{MJ}$):

\begin{equation}
    \Delta E_{\text{electric}} = \frac{48.42}{3.6} \approx 13.45 \, \text{kWh/kg}
\end{equation}

In a realistic engineering scenario, we must account for energy losses due to laser/microwave 
transmission, atmospheric attenuation, and motor efficiency. We assume a total system efficiency
 factor of $\eta = 0.2$ ($20\%$). The actual energy input required from the grid, 
 $E_{\text{input}}$, is:

\begin{equation}
    E_{\text{input}} = \frac{\Delta E_{\text{electric}}}{\eta} = \frac{13.45}{0.2} = 67.25 \, \text{kWh/kg}
\end{equation}

Given an electricity price of $P = 0.05 \, \text{USD/kWh}$, the cost to lift 1 kg of payload is:

\begin{equation}
    \text{Cost} = E_{\text{input}} \times P = 67.25 \times 0.05 \approx 3.36 \, \text{USD}
\end{equation}

Thus, under conservative efficiency assumptions, the energy cost to transport supplies to the Apex Anchor 
is approximately \textbf{\$3.36 per kilogram}, which is \textbf{\$3360 per metric ton}.

Considering the ratio of \textbf{\ 1:4.2} we concluded in 3.1.1, we know that to lift one ton of 
net weight material to the apex anchor, the whole cost would be: 
\begin{equation}
   Cost_{\text{total}} = Cost_{\text{net weight}} + Cost_{\text{fuel}} = 3360 \cdot (1+4.2) = 17472 \text{USD/t}
\end{equation}


%-----------------------------------------
%3.2 火箭的基本情况
%-----------------------------------------
\subsection{Overall Analysis of Rockets}
\subsubsection{Model of Rocket Capacity Growth}













\subsubsection{Model of Rocket Cost Decrease}













%------------------------------------------
%3.3 量化模型引入(Cobb-Douglas形式)
%------------------------------------------

\subsection{Rating System and Explanation}







% 函数形式-----------------------------
\subsubsection{Format of the Function}





% 选择此模型用于评分的理由---------------------
\subsubsection{Reasons of Using this Rating Method}




%------------------------------------
% 3.4 基本物流原则:遵循S型建设曲线
%------------------------------------
\subsection{"S" Shape Construction Curve}







