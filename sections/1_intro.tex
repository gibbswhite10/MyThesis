%!TEX root = ../main.tex
\section{Introduction}
\subsection{Problem Background}
A Moon Colony is definetly a great leap for mankind to explore and make use of the universe.
Though the task of building a colony seems samll compared with the whole process of exploring
universe, this could be extremely difficult and costly starting from 2050.

With the aid of space elevator system and corresponding facilities like Galactic harbours and 
apex anchors, we're able to combine this efficient new system with traditional methods of launching
rockets from earth to jointly contribute to the construction of the Moon Colony which is designed 
to hold 100,000 permanent residents.

To aid the MCM Agency in navigating these complex choices, we developed mathematical 
models to simulate and compare these transportation paradigms. By quantifying the 
trade-offs between the Space Elevator's consistent lift capacity and the flexible 
but expensive nature of traditional rockets, our study provides a precise 
roadmap for optimal supply chain management. %此处是ai写的

\subsection{Restatement of the Problem}
Based on the background mentioned above, our goal is to develop a comprehensive 
model that assists the MCM Agency in determining the most cost-effective, timely, 
and sustainable strategy for building and maintaining the Moon Colony. The model 
should address the following questions:

\begin{itemize}
    \item \textbf{Scenario Analysis:} Determine the optimal logistics strategy 
    among three scenarios: utilizing the Space Elevator System exclusively, relying 
    solely on traditional rocket launches, or adopting a hybrid combination of both.
    
    \item \textbf{Reliability under Uncertainty:} Identify to what extent the  
    solutions change when the transportation systems are not in perfect working 
    condition or even damaged for a period, accounting for factors such as tether 
    swaying or rocket launch failures. % ai写的
    
    \item \textbf{Resource Sustainability:} Investigate the water supply requirements 
    for the 100,000-person colony over a one-year period and evaluate the additional 
    logistical burden and costs required to ensure survival. %ai写的
    
    \item \textbf{Environmental Impact:} Discuss the ecological impacts of each 
    scenario on Earth's environment, specifically focusing on possible atmospheric pollution, 
    and propose adjustments to minimize these impacts.
\end{itemize}

\subsection{Our Work}
To address these challenges, we constructed a multi-stage modeling framework:

\begin{itemize}
    \item \textbf{In Task 1}, we develop a general rating system based on \textbf{Cobb-Douglas
     Function} considering 2 input factors: construction period and total cost. The rating system's
     output indicates the combination of economic feasibility and acceptability of construction
     span. Large amounts of simulation data are input into the rating system and thus we derive
     a best solution in the hypothetical perfect scenario.

    \item \textbf{In Task 2}, we employ a \textbf{Stochastic Reliability Model} 
    incorporating Monte Carlo simulations to address system imperfections. By 
    introducing random variables for operational disruptions (e.g., elevator 
    downtime or launch window delays), the model evaluates the robustness of 
    the proposed schedules and costs. % ai写的
    
    \item \textbf{In Task 3}, we develop a \textbf{Metabolic Demand Model} to 
    quantify the colony's water needs. This model calculates the daily per capita 
    consumption and recycling efficiency to predict the cumulative mass of water 
    that must be transported annually after the colony is inhabited. % ai写的,模型名有待考虑
    
    \item \textbf{Finally}, we conduct an \textbf{Environmental Impact Assessment 
    } to compare the carbon emmision of rocket propellants against the 
    clean energy profile of the Space Elevator. We propose a "Green Logistics 
    Index" to identify the strategy that balances construction efficiency with 
    ecological preservation.
    %%%此处 OUR WORK 可以用流程图来表示!%%%
\end{itemize}




























